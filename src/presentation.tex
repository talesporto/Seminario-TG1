%% LaTeX Beamer presentation template (requires beamer package)
%% see http://latex-beamer.sourceforge.net/
%% idea contributed by H. Turgut Uyar
%% template based on a template by Till Tantau
%% this template is still evolving - it might differ in future releases!

\documentclass{beamer}
\usepackage[brazil]{babel}
\usepackage[utf8]{inputenc}
\usepackage{amsfonts}
\usepackage{amsmath}

\mode<presentation>
{
    \usetheme{Boadilla}
    
    \setbeamercovered{transparent}
}


\title{Reconhecimento e localização de pessoas em um SmartSpace}
%\subtitle{}

% - Use the \inst{?} command only if the authors have different
%   affiliation.
\author{Tales Mundim Andrade Porto}
\author{Danilo Ávila Monte Christo}
%\author{\inst{1}}

% - Use the \inst command only if there are several affiliations.
% - Keep it simple, no one is interested in your street address.
\institute[UnB]
{
    %\inst{1}%
    Instituto de Ciências Exatas\\
    Departamento de Ciência da Computação\\
    Universidade de Brasília
}

\date{20 de junho de 2011}


% This is only inserted into the PDF information catalog. Can be left
% out.
%\subject{Talks}



% If you have a file called "university-logo-filename.xxx", where xxx
% is a graphic format that can be processed by latex or pdflatex,
% resp., then you can add a logo as follows:

% \pgfdeclareimage[height=0.5cm]{university-logo}{university-logo-filename}
% \logo{\pgfuseimage{university-logo}}



% Delete this, if you do not want the table of contents to pop up at
% the beginning of each subsection:
%\AtBeginSubsection[]
%{
%\begin{frame}<beamer>
%    \frametitle{Sumário}
%    \tableofcontents[currentsection,currentsubsection]
%    \end{frame}
%}

% If you wish to uncover everything in a step-wise fashion, uncomment
% the following command:

%\beamerdefaultoverlayspecification{<+->}

\begin{document}

% ------------- TITLE PAGE -------------
\begin{frame}
\titlepage
\end{frame}
% ------------- TITLE PAGE -------------


% ------------- SUMARIO -------------
\begin{frame}
\frametitle{Sumário}
\tableofcontents
% You might wish to add the option [pausesections]
\end{frame}
% ------------- SUMARIO -------------


\section{Introdução}

% ------------- X -------------
\subsection{Contexto}
\begin{frame}
    \frametitle{Redes sem fio no dia a dia}
    Redes sem fio são amplamente utilizadas no dia a dia.
    
    \begin{figure}[h]
    \centering \includegraphics[scale=0.6]{img/wireless_dia_a_dia.jpg}
    \caption{Redes sem fio no dia a dia}
    \label{wireless_dia_a_dia} 
    \end{figure}
\end{frame}

\begin{frame}
    \frametitle{Rede sem fio residencial}
    
    \begin{figure}[h]
    \centering \includegraphics[scale=0.6]{img/wireless_network.jpg}
    \caption{Exemplo de rede sem fio residencial}
    \label{wireless_residencial} 
    \end{figure}
\end{frame}

\begin{frame}
    \frametitle{Rede Ad-Hoc}
    
    \begin{figure}[h]
    \centering \includegraphics[scale=0.4]{img/adhoc.jpg}
    \caption{Exemplo de rede sem fio ad-hoc}
    \label{wireless_ad_hoc}
    \end{figure}
\end{frame}

\begin{frame}
    \frametitle{Rede Ad-Hoc}
    
    \begin{figure}[h]
    \centering \includegraphics[scale=0.9]{img/adhoc_militar.jpg}
    \caption{Uso militar em áreas sem infra-estrutura}
    \label{adhoc_militar} 
    \end{figure}
\end{frame}
% ------------- C -------------


\section{Problema}
\begin{frame}
    \frametitle{Problema}
    \pause Escolher um grupo $n$ dado um conjunto $N$ de nós, em que ($n$
    $\leqslant$ $N$), de modo que este subgrupo possa executar uma determinada ação num
    determinado momento.\\
    \pause Esta seleção deve ser feita de maneira eficiente!\\
    \pause O que quer se dizer por eficiente?\\
    \pause Seleção rápida, seleção justa, economiza de energia!
\end{frame}

\section{Justificativa}
\begin{frame}
    \frametitle{Justificativa}
    \begin{itemize}
      \item Aumento no rendimento dos dipositivos, já que haverá economia de
      energia
      \item Diminui-se os custos, já que pode-se tirar o elemento centralizador
      da infraestrutura.
    \end{itemize}
\end{frame}


% ------------- SOBRE O TRABALHO -------------
\section{Sobre o Trabalho}
\begin{frame}
    \frametitle{Sobre o Trabalho}
    \begin{itemize}
        \item Como será executado?
        \item Qual a metodologia?
    \end{itemize}
\end{frame}
% ------------- SOBRE O TRABALHO -------------


\subsection{Problema}
\begin{frame}
    \frametitle{Problema}
    Escolher um grupo $n$ dado um conjunto $N$ de nós, em que ($n$ $\leqslant$
    $N$), de modo que este subgrupo possa executar uma determinada ação num determinado momento.
\end{frame}


\subsection{Hipótese}
\begin{frame}
    \frametitle{Hipóteses}
    \begin{itemize}
      \item É possível encontrar uma política de seleção que seja rápida e
      maximize a economia de energia, sem prejudicar a justiça e a execução da
      ação(por exemplo, transmissão de dados).
      \item Cenário:
        \begin{itemize}
            \item Tem-se $N$ nós
            \item Temos um subgrupo de $n$ nós que podem executar determinada
            ação num determinado tempo
            \item Este grupo não é necessariamente único
            \item Sujeito a problemas de identificação dos nós
        \end{itemize} 
    \end{itemize}
\end{frame}


\subsection{Objetivos e Resultados Esperados}
\begin{frame}
    \frametitle{Objetivos e Resultados Esperados}
    \begin{itemize}
      \item \textbf{Objetivo Geral}
        \begin{itemize}
          \item Analisar os métodos existente e propor uma solução de política
          eficiente
        \end{itemize}
      \item \textbf{Objetivo Específico}
        \begin{itemize}
          \item Propor uma solução eficiente e validá-la através de um simulador
          que será desenvolvido.
        \end{itemize}
      \item \textbf{Resultados Esperados}
        \begin{itemize}
          \item Proposta de política que:
            \begin{itemize}
              \item seja maximizado a economia de energia
              \item seja rápida
              \item seja justa
              \item seja maximizado a vazão de dados
            \end{itemize}
        \end{itemize}
    \end{itemize}
\end{frame}


\subsection{Metodologia}
\begin{frame}
    \frametitle{Metodologia}
    \begin{itemize}
    \pause \item Levantamento do estado da arte
        \begin{itemize}
            \item Identificação de problemas e pontos em aberto
        \end{itemize}
    \pause \item Proposta de solução
    \pause \item Desenvolvimento de um simulador
    \pause \item Validar solução
        \begin{itemize}
            \item Utilizando o simulador, executar coleta de dados para análise
                e validação da proposta de solução para uma política de seleção
        eficiente
        \end{itemize}
    \end{itemize}
\end{frame}


\subsection{Cronograma}
\begin{frame}
    \frametitle{Cronograma}
    \begin{itemize}
        \item Julho: Levantamento do estado da arte
        \item Agosto: Levantamento do estado da arte
        \item Setembro: Desenvolvimento do simulador
        \item Outubro: Término do desenvolvimento e coleta de dados
        \item Novembro: Análise de resultados e aspectos finais
        \item Dezembro: Apresentação
    \end{itemize}
\end{frame}


% ------------- REFERENCIAS -------------

\nocite{Stevenson09,Cordeiro04,Cormen94,Kurose2005,Stevens94,Tanenbaum03,Zimmermann80,Bae09,Jung08,Park97}

\section{Referências}

\frame[allowframebreaks]{
  \frametitle{Referências}
  \bibliographystyle{plain}
  \bibliography{bibliografia}
}

\begin{frame}
    \frametitle{ }
    \centerline{Obrigado!}
\end{frame}

\end{document}
